\documentclass[5p,authoryear]{elsarticle}

\usepackage{graphicx}

% The amssymb package provides various useful mathematical symbols
\usepackage{amssymb}

\newcommand{\itbold}[1]{\textbf{\textit{#1}}}
\newcommand{\pdif}[2]{\frac{\partial #1}{\partial #2}}
\newcommand{\odif}[2]{\frac{d #1}{d #2}}
\newcommand{\rotr}[1]{\nabla_{\bf r} \times #1}
\newcommand{\rot}[1]{\nabla \times #1}
\newcommand{\divr}[1]{\nabla_{\bf r} \cdot \textbf{\textit{#1}}}
\renewcommand{\div}[1]{\nabla \cdot #1}

\begin{document}

\title{PSDF: Particle Stream Data Format for N-Body Simulations}

\author[WMF]{Will M. Farr\corref{cor1}}
\ead{w-farr@northwestern.edu}
\author[JA]{Jeff Ames}
\author[PH]{Piet Hut}
\author[JM]{Junichiro Makino}
\author[SM]{Steve McMillan}
\author[TM]{Takayuki Muranushi}
\author[KN]{Koichi Nakamura}
\author[KN2]{Keigo Nitadori}
\author[SPZ]{Simon Portegies Zwart}

\cortext[cor1]{Corresponding author}

\address[WMF]{Northwestern University Center for Interdisciplinary
  Research in Astrophysics, 2145 Sheridan Rd., Evanston IL 60208 USA}

\address[JM]{Interactive Research Center of Science, Graduate
  School of Science and Engineering Tokyo Institute of Technology,
  2--12--1 Ookayama, Meguro, Tokyo 152-8551, Japan}

\address[PH]{Institute for Advanced Study, Princeton, NJ 08540, USA}

\begin{abstract}
  We present a data format for the output of general N-body
  simulations, allowing the presence of individual time steps.  By
  specifying a standard, different N-body integrators and different
  visualization and analysis programs can all share the simulation
  data, independent of the type of programs used to produce the data.
  Our Particle Stream Data Format, PSDF, is specified in YAML, based
  on the same approach as XML but with a simpler syntax.  Together
  with a specification of PSDF, we provide background and motivation,
  as well as specific examples in a variety of computer languages.  We
  also offer a web site from which these examples can be retrieved, in
  order to make it easy to augment existing codes in order to give
  them the option to produce PSDF output.
\end{abstract}

\begin{keyword}
% keywords here, in the form: keyword \sep keyword
  Stellar dynamics \sep Method: $N$-body simulation
% PACS codes here, in the form: \PACS code \sep code
\end{keyword}

\maketitle

\section{Introduction}

The simplest N-body calculations use a shared time step length for all
particles.  This implies a rather simple structure of the output.
With N particles and k time steps, the output takes on the form of an
$N*k$ matrix of particle data, where the latter typically contain the
mass, position and velocity of a single particle at a specific time,
with possible additional information such as higher derivatives of the
position (acceleration, jerk, etc.), the value of the potential at the
position of that particle, and so on.  The output of this matrix can
be done by ordering in time or by ordering by the identity of
particles, in which case each world line is output separately.

Some complications may occur when particles are removed, for example
because they are escaping from the system, or because they represent a
star that undergoes a destructive supernova leaving no remnant.
However, the basic I/O structure is simple enough that it is easy to
present these kinds of data in one of the standard data formats, such
as FITS or HDF, with a brief description of what is what.

The situation gets vastly more complicated, though, when we allow for
individual time steps.  Simulations of dense stellar systems, such as
open and globular star clusters, as well as galactic nuclei, have
relied on the use of individual time steps very early on, at least
since the 1960s.  The reason is that the presence of close binaries
and triples in such systems would increase the computer power needed
by orders of magnitude in case of shared time steps, compared to
individual time steps.  In addition, cosmological codes, too, are
moving toward the use of individual timesteps, given the increasingly
large discrepancies of intrinsic time scales that come with
increasingly high spatial resolution.

The simplest way to output data from individual time step codes would
be to stick to shared time steps.  Indeed, typical legacy codes, such
as NBODY6, do just that by default.  If all one wants to do is to make
a fixed movie of a simulation run, that approach suffices.  However,
when we interactively inspect the results of a simulation run, we want
to be able to zoom in and out, and speed up and slow down the rate at
which we run the graphics presentation of the run.  With a fixed
initial output rate, it may not be possible to interpolate the motion
of the particles that move at high speeds.  Phrased differently, an
output rate high enough to faithfully present the motion of all
particles may be prohibitively expensive in terms of memory.  It would
be much better to let the graphics program itself decide how and where
to extrapolate, given the original data it has received from a
simulations code.

For example, when we display the dense center of a star cluster, the
graphics program can then use the full information for the rapidly
moving particles, while interpolating the data for the slower halo
particles.  Such an approach can easily save orders of magnitude of
memory storage requirement.  An implementation of this approach was
made by Steve McMillan (199? -- {\bf Steve, please provide reference
  and a few-line summary}).  However, this implementation was
handcrafted for a specific code, reflecting the data structure used in
that code.  Clearly, it would be desirable to have a more universal
data format that would allow different codes to share data in a more
transparent way.

Other concerns are to make a data format standard machine independent,
to make allowance for parallel processing, and to avoid serious
overhead penalties with respect to performance ({\bf Jun, do you want
  to add a few lines here?})

In the following sections, we describe a machine-independent,
algorithm-independent data format for storing the results of a
simulation of point-mass gravitational dynamics using individual
timesteps, which we call the ``Particle Stream Data Format,'' or PSDF.

\section{Basic idea}

We wish to store the evolving state of a gravitating system of $N$
bodies throughout a simulation with individual timesteps for each
particle. Conceptually, what we need is a stream of phase-space
information of particles, updated each time the integration algorithm
adjusts a particle's phase space information.  One possibility for
such a stream could be:
\begin{verbatim}
  particle_id, time, mass, x, y, z, vx, vy, vz, ...
  particle_id, time, mass, x, y, z, vx, vy, vz, ...
  particle_id, time, mass, x, y, z, vx, vy, vz, ...
\end{verbatim}
However, the data format should be flexible enough to be able to
include more information, if available, such as
\begin{itemize}
\item radius, and other info related to stellar evolution
\item merger history
\item fluid properties if a particle is an SPH particle
\item {\bf other things?}
\end{itemize}

One way to construct such a flexible data format is to use
self-describing data format, such as XML or YAML. For simplicity, we
adopt YAML \citep{YAML2011} here; there are libraries for reading and
writing YAML in many popular programming languages, and the format is
simple enough to be understood easily by humans, even if they are not
already familiar with it.

\subsection{Some basics of YAML}

The following is a simple example of data in YAML format.
\begin{verbatim}
--- !Particle
id: 0
r:
  - 0.1
  - 0.2
  - 0.3
v:
  - -1
  - -2
  - -3
m: 1.0
\end{verbatim}
In the above example, the line
\begin{verbatim}
--- !Particle
\end{verbatim}
Is the header, which indicates that it describes the data of an object
of type {\tt Particle}.  The line
\begin{verbatim}
id: 0
\end{verbatim}
defines a field with name ``id'', and value 0.  The text
\begin{verbatim}
r:
  - 0.1
  - 0.2
  - 0.3
\end{verbatim}
means the field ``r'' is an array with three elements. The first ``-''
means this line is a data for an array.  By default, numbers without
``.'' are regarded as integers, and with ``.'' floating point. Note
that indentation has meaning here and ``-'' must be indented the same
level or deeper than ``r'' and should be aligned.  The ``v'' and ``m''
fields behave similarly.  The order of the fields is not important;
the {\tt Particle} object behaves as a \emph{map} from names, like
``r'', to values, like the array ``\verb|[0.1, 0.2, 0.3]|''.

\section{Particle Stream Data Format}

With the minimal description of YAML in the previous section, we can
now define the Particle Stream Data Format: the data format is a
stream of YAML representations of particle objects.  For example, a
valid PSDF fragment is
\begin{verbatim}
--- !Particle
id: 0
t: 0
r:
  - 0.1
  - 0.2
  - 0.3
v:
  - -1
  - -2
  - -3
m: 1.0
--- !Particle
id: 1
t: 0
r:
  - 0.2
  - 0.3
  - 0.4
v:
  - 0
  - 0
  - 0
m: 1.0
\end{verbatim}
This fragment describes two particles, at $t = 0$, with ids 0 and 1.

Particle objects behave in YAML as mappings from names to values.
Therefore, a specification of the meaning of certain names and a
procedure for handling unknown names in the stream are sufficient to
define the data format.  In Table \ref{tab:names} we list the reserved
names of our PSDF.  Any particular particle in a PSDF stream need not
include a value in its map for any of these names, but if it does, the
value must have the meaning in the table; similarly, if a particle
object in a PSDF stream does contain a value with one of the meanings
in the table, then it should be identified by the corresponding name.
Note that these requirements allow for easy extension of a PSDF stream
with application-specific information by including any names and
values not in Table \ref{tab:names} needed by the specialized
application.  Programs that understand this additional information can
benefit, while those that do not will still be able to function using
the basic information from any included values of Table
\ref{tab:names}.  In the future, we intend to provide extensions that
are useful for hierarchical decomposition of an $N$-body system and
for description of fluid SPH particles.

\begin{table}
\begin{tabular}{|c|l|}
\hline
Name & Meaning\\
\hline
id & index (can be arbitrary text)\\
m & mass\\
t & time\\
t\_max & max time to which this record is valid \\
r  & position, array with three elements\\
v  & velocity, array with three elements\\
pot  & potential\\
acc  & acceleration, array with three elements\\
jerk  & jerk, array with three elements\\
snap  & snap, array with three elements\\
crackle  & crackle, array with three elements\\
pop  & pop, array with three elements\\
\hline
\end{tabular}
\caption{\label{tab:names} The reserved names of the PSDF and their
  meanings.  Other names occurring in the data stream are to be
  ignored if they are not meaningful to an application or interpreted
  in an application-specific way if they are.  A particle object in a
  PSDF data stream need not contain any of these names, but if it does
  they must have the meanings above; similarly, if a particle object
  contains data with the above meanings, it should be identified by
  the associated standard name.}
\end{table}

We require that time, position, velocity and higher derivatives are
consistent (for example, if position is given in parsecs and time in
years, velocity must be in parsec/year).  The name {\tt t\_max} is
rather special, in that it gives the maximum possible time that this
record is used to predict the orbit of this particle; we expect that
this may prove useful to prevent invalid extrapolation in programs
that process the PSDF.

% In the a data file, if the record of one particle is not produced
% after certain time, the analysis/visualization program regard that it
% somehow vanished after time {\tt t\_max}.

% We considered the possibility of defining a special record for
% creation and destruction of a particle, but decided against that to
% keep the parser as simple as possible to write.

A complete PSDF object is a stream of particle objects describing the
states of individual particles in the system at particular times.
Such a stream may be consumed as it is produced, as in the case of an
integrator program whose output is directed to a graphics program that
displays the result of the integration; or such a stream may we
written to one or more files to be processed at a later date.  We do
not impose any particular ordering on the particle records in a PSDF
stream.  For some applications an ordering in time may be appropriate,
while for others an ordering in particle id may be better, or even
more complex orderings; later in this paper we provide references to
code that can convert between the time and particle-id orderings.

\section{Rationale}

Our goal is to describe a data format that is 
\begin{enumerate}
\item Space-efficient for storing the data individual-timestep
  $N$-body simulations.
\item Simple enough to be human readable and writeable.
\item Information-rich.  It should allow for post-processing and
  analysis or even continuation or re-running of a simulation.
\item Flexible enough to accommodate the special needs of programs
  that have more complex objects than point-mass particles.
\item Simple and safe to read and parse within a program.
\item Composable, so that fragments of the data can be split off for
  separate analysis and recombined easily.
\end{enumerate}
In this section we describe how these design goals led to the
specification in the last section.

\subsection{Space Efficient and Human-Readable}

As outlined in the introduction, it is extremely wasteful to produce a
complete system snapshot in an $N$-body simulation every time some of
the bodies update their positions or velocities.  The PSDF format
allows for the output of only the changed data---the new states of the
updated particles at the new time---to the stream.  Though the native
format is text-based, for readability, we have found that common
compression algorithms such as {\tt gzip} applied to files containing
PSDF data produce output that is within 10\% of the size of equivalent
compressed binary data.

\subsection{Information-Rich and Flexible}

The fundamental state of a point-particle can be specified in 8
numbers: one mass, one time, three position coordinates and three
velocity coordinates.  However, some auxiliary information about the
particle's state can be very helpful: accurate prediction of the
particle's position and velocity---for example, to display its track
in a visualization---can be facilitated by information about the
higher derivatives of its position in time.  When available to the
integration routine, these can be easily provided by our format (see
Table \ref{tab:names}).  

Not all particles in many $N$-body simulations are point-masses!  For
example, simulations may attempt to model stellar evolution, and
therefore store ``star'' properties like radii and masses, or entropy
profiles with their particles.  Or, simulations may include fluid
particles subject to non-gravitational forces for SPH calculations.
To attempt to standardize names for every possible particle property
would result in a rigid and cumbersome format; instead, by allowing
arbitrary application-specific names in PSDF particle mappings that
can be ignored when not understood we permit complex applications to
work with application-specific data while ensuring that simple
applications can make use of the parts of the data they
understand.

\subsection{Simple, Safe, and Composable}

By using a general format (YAML) in wide use for our PSDF, we ensure
that there are mature, debugged libraries available for reading and
writing our format.  However, YAML is simple enough that it can we
written and modified easily ``by hand'' in a text editor.  The idea of
streaming updates to individual particle states also meshes nicely
with the evolution algorithms in most $N$-body integrators, making the
format easy to write from within such a code.

PSDF objects are descriptions of \emph{data}, not instructions for
actions for an application.  For example, there is no instruction for
``adding'' or ``deleting'' a particle from the stream.  Adding such
instructions would require applications to implement interpreters for
implementing the instructions in the stream, which raises issues of
security and language design that would significantly complicate the
specification.

Finally, PSDF documents are \emph{composable}, meaning that any two
PSDF streams can be concatenated to form another valid PSDF stream,
and a single stream can be split into any number of valid sub-streams.
It is quite simple to write a short script in any number of languages
that consumes a PSDF stream and produces another recording the history
of a particular particle in the original stream, for example.  Adding
header information, or instructions, or any other meta-data to the
stream defeats this goal by requiring specification of some way to
split and combine the associated meta-data.

\section{Repository}

We have examples of codes that generate and manipulate PSDF streams at 
\begin{verbatim}
https://github.com/jmakino/Particle-Stream-Data-Format
\end{verbatim}
The examples are in various different languages, and include
\begin{itemize}
\item Programs to generate initial conditions for $N$-body simulations
  in PSDF format.
\item Code and references to several different individual-time-step
  integrators that can take PSDF input and advance the corresponding
  system in time, producing PSDF output at each step.
\item Various post-processing tools that compute useful system
  properties from PSDF input.
\item A visualization program that takes PSDF input and produces a 3D
  representation of the system that can be played forward and
  backward, zoomed, etc.
\end{itemize}
We hope that the examples we provide will make it easy for the
community to use PSDF in their simulations, and that these users will,
in turn, contribute their useful programs back to the repository as
examples for future users.

\section*{Acknowledgments}

\bibliographystyle{elsarticle-harv}
\bibliography{data-format}

\end{document}
